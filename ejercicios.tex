\documentclass[11pt,spanish]{article} % Idioma
\usepackage{babel}
\usepackage[T1]{fontenc}
\usepackage{textcomp}
\usepackage[utf8]{inputenc} % Puede depender del instrucción, sistema o editor
\usepackage{wrapfig} % Imagenes
% \graphicspath{ {./imagenes/} }

\usepackage[left=2.75cm,top=2.5cm,right=2cm,bottom=2.5cm]{geometry} % Márgenes
%\usepackage{pstricks} % Gráficas, movilidad, árboles y otros

\usepackage{amssymb, amsmath} % Símbolos matemáticos
\usepackage{amsthm} % Teoremas, lemas, pruebas...
\usepackage{cancel} % Cancelar expresiones
\usepackage{multirow} % Tablas
\usepackage{multicol}
\usepackage{enumitem} % Enumerados a), b), c)... usando \begin{enumerate}[label=\alph*)]

\usepackage{graphicx} % Inserción de imágenes
\usepackage{xcolor} % Colores
\usepackage{color}
\definecolor{gray97}{gray}{.97}
\definecolor{gray75}{gray}{.75}
\definecolor{gray45}{gray}{.45}

\usepackage[hidelinks]{hyperref}  % Enlaces

\usepackage{listings} % Escribir código en diferentes lenguajes de programación
\usepackage{longtable} % para tablas largas
\lstset{ frame=Ltb,
framerule=0pt,
aboveskip=0.5cm,
framextopmargin=3pt,
framexbottommargin=3pt,
framexleftmargin=0.4cm,
framesep=0pt,
rulesep=.4pt,
backgroundcolor=\color{gray97},
rulesepcolor=\color{black},
%
stringstyle=\ttfamily,
showstringspaces = false,
basicstyle=\small\ttfamily,
commentstyle=\color{gray45},
keywordstyle=\bfseries,
%
numbers=left,
numbersep=15pt,
numberstyle=\tiny,
numberfirstline = false,
breaklines=true,
}



\title{Ejercicios de Análisis Matemático II}
\author{ }
\date{\today}

% Comandos
\newcommand{\enunciado}[1]{\vspace{0.75cm} \textbf{#1}}


% % % % % % % % % % % % % % % % % % % % % % % % % % % % % % % % %
%					 Inicio del documento
% % % % % % % % % % % % % % % % % % % % % % % % % % % % % % % % %
\begin{document}


\maketitle
\tableofcontents % Generando el indice
\newpage
\setlength\parindent{0pt} % Quitamos la sangría




\section{Sucesiones de funciones}
\subsection{Sucesiones de funciones}
	% 

	%\enunciado{Dado $\alpha \in \R$, consideremos la sucesión de funciones $\{f_n\}$, donde $\{f_n\}:[0,1] \longleftrightarrow \R$ es la función definida para todo $x\in[0,1]$ por:
$$f_n(x) = n^\alpha x(1-x^2)^n$$} 

El conjunto en el que tenemos que estudiar las funciones es el $[0,1]$, y las funciones son:

$$f_n(x) = n^\alpha x(1-x^2)^n$$ 
$$f_n'(x) = n^\alpha(1-x^2)^n - 2n^{\alpha+1}x^2(1-x^2)^{n-1} = n^\alpha(1-x^2)^{n-1}(1-x^2(1+2n))$$

Tenemos que $f_n '(x) = 0 \iff x=\left\{
\begin{array}{l}
1, \\
\frac{1}{\sqrt{2n+1}}
\end{array}\right.$

Vemos rápidamente que $f_n'(0) > 0$ por lo que las funciones son crecientes en la primera parte, llegan hasta $\frac{1}{\sqrt{2n+1}}$ y luego debe de bajar (porque $f_n(0) = f_n(1) = 0$), así que vamos a definir nuestra sucesión de máximos como $x_n = f_n\left(\frac{1}{\sqrt{2n+1}}\right)$. La estudiamos:

$$x_n = \frac{n^\alpha*(2n)^n}{(2n+1)^{n+\frac{1}{2}}} =  \frac{n^{n+\alpha}}{\sqrt 2(n+\frac{1}{2})^{n+\frac{1}{2}}} = \frac{1}{\sqrt 2}\frac{n^\alpha}{(n+\frac{1}{2})^{\frac{1}{2}}}\left(\frac{n}{n+\frac{1}{2}}\right)^n$$

La fracción de la derecha es equivalente a $e^{n\cdot ln\left(\frac{n}{n+\frac{1}{2}}\right)}$. Esto tiende a $\frac{1}{\sqrt e}$ cuando $n$ tiende a infinito. Usando que solo tenemos que estudiar este límite:

$$\lim\limits_{n->\infty} n\cdot ln\left(\frac{n}{n+\frac{1}{2}}\right)$$

Para hallar el límite estudiamos la siguiente función y aplicamos L'Hôpital:

$$g(x) = \frac{ln\left(\frac{n}{n+{\frac{1}{2}}}\right)}{\frac{1}{n}} = \frac{ln(n)-ln(n+1/2)}{\frac{1}{n}}$$



Sabiendo que la fracción de la derecha converge a $\frac{1}{\sqrt e}$ (habiendo hecho L'Hôpital arriba) y la fracción de la izquierda converge también, solo nos queda mirar la de en medio. Si $\alpha < 1/2$, esa sucesión converge a 0. Si $\alpha > 1/2$ esa sucesión diverge, y en el caso de $\alpha = 1/2$ esa sucesión converge a 1. Nos queda en total que ${x_n}$ converge a 0 $\iff \alpha < 1/2$. Por lo que hay convergencia uniforme solo en ese caso.

Para el caso en el que tenemos el conjunto $[\rho, 1]$ para algún $\rho > 1$, esta vez la funcion $f_n$ será decreciente en todo el intervalo a partir de un determinado $n$ (ya que el punto donde se alcanzaba el máximo antes se acerca a 0 conforme $n$ se va agrandando). Por lo que esta vez, a partir de ese $n_0$ con el que las $f_n$ con $n>n_0$ son decrecientes podemos coger la sucesión de máximos $x_n = f_n(\rho)$. Pero de nuevo, si hemos hecho la convergencia puntual, esta sucesión converge a 0, por lo que hay convergencia uniforme en este intervalo.

	% 

	%\begin{ejercicio}{%
Para cada $n \in \N$, sea $f_n:]0,\pi[ \to \R$ la función dada por:
\[f_n(x) = \frac{\sin^2(nx)}{n\sin(x)} \quad (0 < x < \pi)\]
Estudia la convergencia puntual de la sucesión puntual $\{f_n\}$ así como la
convergencia uniforme en intervalos del tipo $]0,a], [a,\pi[$ y $[a,b]$ donde $0 < a< b < \pi$.
}{Pablo Baeyens}

\espacio

\noindent\textbf{$\{f_n\}$  converge puntualmente a 0}

Sea $x \in (0,\pi)$. Para todo $n \in \N$ tenemos:
\[ 0 \leq \frac{\sin^2(nx)}{n\sin(x)} \leq \frac{1}{n\sin(x)} \]
Por tanto, como $\{\frac{1}{n\sin(x)}\} \to 0$, por el teorema del sándwich, la
sucesión converge a 0.

\espacio

\noindent\textbf{La sucesión no converge uniformemente en intervalos de la forma $(0,a]$}

La convergencia uniforme es equivalente a que para cualquier sucesión $\{a_n\}$:

\[\{f_n(a_n) -f(a_n)\} = \left\{\frac{\sin^2(na_n)}{n\sin(a_n)}\right\} \to 0\]

Sea $a_n = \frac{1}{n}$:

\[ \left\{\frac{\sin^2(na_n)}{n\sin(a_n)}\right\}  =
\left\{\frac{\sin^2(1)}{\frac{\sin(\frac{1}{n})}{\frac{1}{n}}}\right\} \to \sin^2(1) \neq 0 \]

\espacio

\noindent\textbf{La sucesión no converge uniformemente en intervalos de la forma $[a,\pi)$}

Sea $a_n = \pi - \frac{1}{n}$:

\[ \left\{\frac{\sin^2(na_n)}{n\sin(a_n)}\right\}  =
\left\{\frac{\sin^2(n\pi -1)}{\frac{\sin(\pi -\frac{1}{n})}{\frac{1}{n}}}\right\} =
\left\{\frac{\sin^2(1)}{\frac{\sin(\frac{1}{n})}{\frac{1}{n}}}\right\} \to \sin^2(1) \neq 0 \]

Ya que $\sin(x) = \sin(\pi-x)$.

\espacio

\noindent\textbf{La sucesión converge uniformemente en intervalos de la forma $[a,b]$}

Por el teorema de Weierstrass
$\exists M >0: \forall x \in [a,b]: \left|\frac{1}{\sin(x)}\right| \leq M$. Por tanto:

\[ \left|\frac{\sin^2(nx)}{n\sin(x)}\right| \leq \frac{M}{n} \]

Como $\{\frac{M}{n}\} \to 0$, la sucesión converge uniformemente en intervalos
de la forma $[a,b]$.

\end{ejercicio}
 

	% 

	% 

	% 

	%\enunciado{
Sea, para cada $n\in\mathds{N}$,
$$f_n(x) = \frac{x}{n^\alpha(1+nx^2)} \hspace{0.5cm} (x\geq0)$$
Prueba que la serie $\sum f_n$ converge}

\textit{a)} puntualmente en $\mathds{R}^+_0$ si $\alpha > 0$
\textit{b)} uniformemente en semirrectas cerradas que no contienen al cero
\textit{c)} uniformemente en $\mathds{R}^+_0$ si $\alpha > 1/2$

\textit{a)} 
Podemos escribir las funciones $f_n$ como:

$$f_n(x) = \frac{x}{n^\alpha+n^{\alpha+1}x^2} $$

Ahora, fijado un $x\in\mathds{R^+}$, usamos el criterio de comparación por paso al límite con la serie $\frac{1}{n^{\alpha+1}}$, que sabemos que converge a un número real, puesto que el exponente de la $n$ en el denominador es mayor estricto que 1:

$$\lim\limits_{n->\infty} \frac{\frac{x}{n^\alpha+n^{\alpha+1}x^2}}{\frac{1}{n^{\alpha+1}}} = \lim\limits_{n->\infty} \frac{x}{\frac{1}{n}+x^2} = \frac{1}{x} $$

Como $1/x$ es un número real distinto de 0, por el criterio sabemos que nuestra serie con los términos $f_n(x)$ también converge (le pasa lo mismo que con la que la hemos comparado).

Para el caso en que $x$ sea 0, es fácil ver que la serie converge puntualmente, puesto que vale 0.


\textit{b)}  Sea la semirrecta $[c, \infty)$, con $c > 0$ podemos ver que:

$$f_n(x) = \frac{x}{n^\alpha+n^{\alpha+1}x^2} \leq \frac{x}{n^{\alpha+1}x^2} = \frac{1}{n^{\alpha+1}x} \leq \frac{1}{n^{\alpha+1}c}$$
Esta última serie es independiente de las $x\in [c, \infty)$ y converge (se puede comprobar por comparación paso al límite con $\frac{1}{n^{\alpha+1}}$ el límite es $c$, que es una constante mayor que 0). Por el test de Weierstrass, podemos afirmar que $f_n$ converge uniformemente en toda la semirrecta $[c, \infty)$.


\textit{c)} Hacemos las derivadas de $f_n$:

$$f_n'(x) = \frac{n^\alpha + n^{\alpha+1}x^2-2x^2n^{\alpha+1}}{(n^\alpha+n^{\alpha+1}x^2)^2} = \frac{n^\alpha(1 - nx^2)}{(n^\alpha+n^{\alpha+1}x^2)^2}$$

Las funciones $f_n$ son crecientes hasta $x=\frac{1}{\sqrt{n}}$, donde empiezan a decrecer, así que esta abscisa es un máximo de las $f_n$. De nuevo, podemos acotar la serie por una sucesión de números reales $a_n$ y concluir que covergen uniformemente en $\mathds{R}^+_0$. Definimos la sucesión de los máximos de la función: $a_n = f_n\left(\frac{1}{\sqrt{n}}\right)$. Vemos que:

$$f_n(x) \leq a_n = f_n\left(\frac{1}{\sqrt{n}}\right) = \frac{\frac{1}{\sqrt{n}}}{n^\alpha(1+1)} = \frac{1}{2n^{\alpha+1/2}} \hspace{0.5cm} \forall x \in \mathds{R}^+_0$$
La serie de términos $a_n$ converge ya que $(\alpha + 1/2) > 1$, así que por el test de Weierstrass, la serie $f_n$ converge uniformemente en $\mathds{R}^+_0$.
	
	% 

	% 

	% 

	% 

	
\subsection{Series de potencias}
	% 

	% 

	% 

	% 

	% 

	% 

	% 

	% 

	% 

	
\newpage
\section{Integral de Lebesgue}
\subsection{Medida de Lebesgue en $\mathbb{R}^N$}
	% 

	% 

	%\enunciado{Sea $(\Omega ,\mathcal{A})$ un espacio medible y sea $\Omega '$ un nuevo conjunto. Sea $f: \Omega \longrightarrow \Omega '$ una aplicación. Probar que $( \Omega ', \{ B\in\Omega ':\ f^{-1}(B)\in \mathcal{A} \} )$ es un espacio medible.}

Definimos $\mathcal{A}'=\{ B\in\Omega ':\ f^{-1}(B)\in \mathcal{A} \}$, debemos demostrar que es $\sigma$-álgebra.

\textit{i)}
$\Omega ' \in \mathcal{A}'$:

$f^{-1}(\Omega ') = \Omega\in\mathcal{A}$ por ser $\mathcal{A}$ $\sigma$-álgebra.

\textit{ii)}
Si $\{A_n\}$ es una sucesión de elementos de $\mathcal{A}'$, entonces $\left( \bigcup_{n\in\mathbb{N}}A_n \right) \in \mathcal{A}'$:

Si $A_i\in \mathcal{A}' \ \forall       
 i\in\mathbb{N} \implies$
$\forall i\in\mathbb{N} A_i\subseteq\Omega '\ : \left( f^{-1}(A_i)\right) \in \mathcal{A}$

\[ \bigcup_{n \in \mathbb{N} }f^{-1}(A_n) \in \mathcal{A} \ por \ ser \ \sigma -álgebra\] 

\textit{iii)}
Si $A\in\mathcal{A}'$ entonces $A^c=\Omega '\setminus A \in \mathcal{A}'$:

Si $A\in \mathcal{A}' \implies f^{-1}(A)\in \mathcal{A}$

\[f^{-1}(A^c) = f^{-1}(\Omega'\setminus A) = f^{-1}(\Omega') - f^{-1}(A) = \Omega - f^{-1}(A) \]

Por ser $\Omega , f^{-1}(A)\in \mathcal{A} \implies f^{-1}(A^c)\in \mathcal{A}$ por ser $\sigma$-álgebra. Por tanto $A^c\in \mathcal{A}$

$\Longrightarrow (\Omega', \mathcal{A}')$ es un espacio medible. 
	% \enunciado{Sea $\mu^*$ una medida exterior en un conjunto no vacío $\Omega$. Probar que la restricci\'on de $\mu^*$ a la $\sigma$  -álgebra $ C_{\Omega, \mu^*}$ es una medida completa ( esto es, todo subconjunto B de un conjunto $Z \in   C_{\Omega, \mu^*}$ tal que $ \mu^* (Z) = 0$ es también un conjunto de la propia $\sigma$ -álgebra)}

Sea Z un conjunto tal que $\mu^* (Z) = 0 $ y $B \subseteq Z$ , entonces , B ser\'a  medible si y solo si 

\[\forall A \subseteq \Omega , \mu^* (A) = \mu^* (A\cap B) + \mu^* (A \cap  B^c).\]

Para probarlo partimos de la desigualdad que nos da la subaditividad de una medida exterior, es decir:
\[A \subseteq \Omega \; \mu^* (A) \leq \mu^* (A \cap B) + \mu^* (A\cap B^c)\]
Si además usamos que\: 
\[A \cap B \subseteq A \cap  B \subseteq Z \Rightarrow  \mu^* (A \cap  B) = 0\]
 entonces obtenemos 
\[A \subseteq \Omega \; \mu^* (A) \leq \mu^* (A \cap  B) + \mu^* (A\cap B^c) \leq  \mu^* (A)\] como queríamos demostrar. 
	
\textbf{5. Probar que M es la mayor $\sigma$ -\'algebra que contiene los intervalos acotados y sobre la que $\lambda ^*$ es aditiva.}



Supongamos que existe otra $\sigma$ -\'algebra $N$ que contiene los intervalos acotados y sobre la que $\lambda ^*$ es aditiva. Terminaremos demostrando que en ese caso $N \subseteq M$.

Recordemos que $M=\{B \cup Z : B\in \mathfrak{B}, \lambda ^*(Z)=0\} \subseteq C_{\mathbb{R},\lambda }$

La $\sigma$-subaditividad nos decía:
\[\lambda ^*(A\cup B) \leq \lambda ^* (A)+ \lambda ^*(B)\]

Llamaremos $\lambda '=\lambda	^* / N$

Por ser una $\sigma$-\'algebra $\Omega \in N$, al estar hablando de intervalos $\Omega = \mathbb{R}$

Sea $E\subseteq \mathbb{R}$, cogemos un conjunto arbitrario $A\subseteq \mathbb{R}$.
Sabemos, en virtud de la propiedad de regularidad de la medida exterior (Prop. 2.1.10), que existe un boreliano $B$

\[ B:A\subseteq B, \lambda '(A)=\lambda '(B) \]

$\lambda '(A)=\lambda '(B)$,  usando la propiedad de que $N$ contiene los intervalos acotados podemos usar la $\sigma-$aditividad. 
$B\cap E, B\cap E^c \in N$ 

$\lambda '(B) = \lambda'\left( (B\cap E) \cup (B\cap E^c) \right) $
$\geqslant\lambda '(A\cap E) + \lambda'(A\cap E^c) \geqslant \lambda '(A)$

Por tanto $E \in C_{\mathbb{R},\lambda '}$ lo que es equivalente a $E \in M$
$\implies N \subseteq M$
 

	% 

	
\textbf{7. Existencia de conjuntos no medibles}

\begin{enumerate}[label=\alph*)]
	\item Probar que la familia $\{x + \mathbb Q : x \in \mathbb R \}$ es una partición de $\mathbb R$. \\
	
	Sea $x \in \mathbb R$. Entonces $x \in x+\mathbb Q$ dado que $x = x + 0$. Por ello, $\displaystyle \bigcup_{x \in \mathbb R} \{x+\mathbb Q\} = \mathbb R$.
	
	Sean $x,y,t \in \mathbb R : t \in x+\mathbb Q$ y $t \in y + \mathbb Q$. Entonces $\exists q_1, q_2 \in \mathbb Q : t = x + q_1 = y + q_2$. Así, $x = y + q_2 - q_1$, y, como $q_2 - q_1 \in \mathbb Q$, $x \in y + \mathbb Q$ y $x + \mathbb Q = y + \mathbb Q$.
	
	Así, esta familia está formada por conjuntos disjuntos (si un elemento está en dos elementos de la familia, estos son el mismo) cuya unión es $\mathbb R$: es una partición de $\mathbb R$.
	
	\item Pongamos $\{x+\mathbb Q : x \in \mathbb R\} = \{A_i : i \in I\} \ (A_i \ne A_j$ para $i \ne j)$ y, para cada $i \in I$, sea $x_i \in A_i \ \cap \ ]0, 1]$. Probar que el conjunto $E = \{x_i : i \in I\}$ no es medible. \\
	
	Sea $\{q_n : n \in \mathbb N\}$ una numeración de $]-1, 1] \cap \mathbb Q$.
	
	Supongamos que $E$ es medible. En tal caso, $\lambda(E) = \lambda(E+k) \ \forall k \in \mathbb R$ dado que $\lambda$ es invariante por traslación. Debido a la $\sigma$-aditividad de $\lambda$ y a que los conjuntos $q_n + E$ son disjuntos entre sí [proof needed], resulta que:
	
	$$\lambda(\bigcup_{n = 1}^{+\infty} (q_n + E)) = \sum_{n=1}^{+\infty}\lambda(q_n + E) = \sum_{n=1}^{+\infty} \lambda(E)$$
	
	Como $\displaystyle ]0, 1] \subseteq \bigcup_{n=1}^{+\infty} (q_n + E) \subseteq \ ]-1, 2]$ [proof needed], también tendremos que $\displaystyle \lambda(]0, 1]) = 1 \le \lambda(\bigcup_{n=1}^{+\infty} (q_n + E)) = \sum_{n=1}^{+\infty} \lambda(E) \le \lambda(]-1, 2]) = 3$. Como esto es imposible tanto si $\lambda(E) = 0$ (en cuyo caso $\displaystyle \sum_{n=1}^{+\infty} \lambda(E) = 0 \ngeq 1$) como si $\lambda(E) \in \mathbb R^+$ (en cuyo caso $\displaystyle \sum_{n=1}^{+\infty} \lambda(E) = +\infty \nleq 3$), la suposición de que $E$ es medible resulta haber sido incorrecta, y $E$ no es medible.
	
	\item Probar que cualquier subconjunto medible de $E$ tiene medida cero. \\
	
	Los conjuntos $q_n + A$ son, de nuevo, disjuntos. Por ello, vuelve a ocurrir que $\displaystyle \lambda(\bigcup_{n = 1}^{+\infty} (q_n + A)) = \sum_{n=1}^{+\infty} \lambda(A)$. De nuevo, $\displaystyle \bigcup_{n=1}^{+\infty} (q_n + A) \subseteq \ ]-1, 2]$ y por ello $\displaystyle \sum_{n=1}^{+\infty} \lambda(A) \le \lambda(]-1, 2]) = 3$. La única posibilidad es que $\lambda(A) = 0$.
	
	\item Sea $M \subseteq \mathbb R$ con $\lambda^*(M) > 0$. Probar que $M$ contiene un subconjunto no medible.
	
	Si $M$ no es medible, el enunciado es trivial ($M$ sería un subconjunto no medible de $M$). Sea $M$ medible, es decir, $\lambda(M) = \lambda^*(M)$. Se observa que $\displaystyle M = \bigcup_{q \in \mathbb Q} M \cap (q + E)$ [proof needed].
	
	Supongamos que $M\cap (q + E)$ es medible para todo $q \in \mathbb Q$. En ese caso: (aparece un $\le$ porque la unión no es disjunta)
	$$\displaystyle \lambda(M) = \lambda(\bigcup_{q \in \mathbb Q} M \cap (q + E)) \le \sum_{q \in \mathbb Q} \lambda(M \cap (q+E))  = \sum_{q \in \mathbb Q} \lambda((M-q) \cap E)$$
	
	Que es igual a $0$ por ser la suma de las medidas de subconjuntos de $E$ medibles (porque suponemos que todos son medibles), las cuales son $0$ por lo probado en $c)$. Contradicción (hemos obtenido que $\lambda(M) \le 0$), por lo cual alguno de los $M\cap (q + E)$ no será medible.
\end{enumerate}
 

	%\enunciado{Probar que la existencia de conjuntos no medibles equivale a la no aditividad de $\lambda^*$.}

 
Sabemos que $E\in M \Longleftrightarrow E \in C_{\mathbb{R}, \lambda^*}$

Queremos probar:
$E \in M \ no \ medible \Longleftrightarrow \lambda^*(A) \geq \lambda^*(A\cap E) + \lambda^*(A \cap E^c) \ \forall A\subset \mathbb{R}^N$

$\Longrightarrow$

Si $E \in \mathbb{R}^N no es medible$, entonces existe $A\subseteq \mathbb{R}^N$ tal que 
\[\lambda^*( (A\cap E) \cup (A\cap E^c) ) = \lambda^*(A) < \lambda^*(A\cap E) + \lambda^*(A\cap E^c)\]
y $\lambda^*$ no es aditiva.

$\Longleftarrow$

Supongamos que $\lambda^*$ no es aditiva. Sean $A,B\subseteq \mathbb{R}^N, \ A\cap B = \varnothing$ tales que

\[\lambda^*(A\cup B) < \lambda^*(A) + \lambda^*(B)\]

Se tiene que
\[ \lambda^*(A\cup B) < \lambda^*( (A\cup B)\cap A ) + \lambda^*( (A\cup B)\cap A^c )\]
y el conjunto $A$ no es medible.
	%\enunciado{Sean $A$ un abierto de $\mathbb{R}^N$ y $f:A \rightarrow \mathbb{R}^M$ una función de clase $C^1$ con $N<M$. Probar que $f(A)$ es de medida cero.}

Sea $G$ un abierto de $\mathbb{R}^N$ y sea $f \in C^1(G)$

1. Si $Z \subseteq G(=B \supseteq Ax\{0\})$, $\lambda (Z) = 0$, entonces $\lambda (f(Z)) = 0$

Definimos $B := A x \mathbb{R}^{M-N} \subseteq \mathbb{R}^M$ y $g: B \rightarrow \mathbb{R}^M$ por
\[ g(x,y) = f(x) \ \forall (x,y)\in B\]
El conjunto $B$ es un abierto de $\mathbb{R}^M$ y $g\in C^1(B)$

El conjunto $Ax\{0\} \subset B$ es de medida cero en $\mathbb{R}^M$ por estar contenido en un hiperplano.

\[\lambda(Ax\{0\}) = \lambda (A) \cdot \lambda (0) = 0\]

Aplicando la proposición anterior $\lambda( g(A,\{0\}) ) = 0 \implies \lambda (f(A)) = 0$
	% 

	% 

	
\subsection{Integral de Lebesgue en $\mathbb{R}^N$}
	% 

	% 

	% 

	% 

	% 

	% 


\subsection{Teoremas de convergencia}
	%\enunciado{Sea $(\Omega, \mathcal{A})$ un espacio de medida, $\{f_n\}$ una sucesión de funciones medibles y $f,g$ dos funciones medibles. Porbar las siguientes afiirmaciones:}


\textit{a)}
	Si $\{f_n\}$ converge c.p.d. a $f$ y a $g$ c.p.d. entonces $f=g$ c.p.d. 

Definimos
\[A = \{ x\in\Omega / \{f_n(x)\} \rightarrow f(x) \} \hspace{0.5cm}
  B = \{ x\in\Omega / \{f_n(x)\} \rightarrow g(x) \} \hspace{0.5cm}
\]

\[  D = \{ x\in\Omega / f(x) = g(x) \} 
\]
Sabemos por hipótesis que $\mu (A^c) = \mu (B^c) = 0$ y si $x\in (A\cap B) \implies f(x) = g(x)$.

\[D^c \subseteq (A\cap B)^c = A^c \cup B^c \]
Suponemos que $D^c$ es medible, por tanto
$\mu (D^c) \leq \mu (A^c \cup B^c) \leq \mu(A^c) + \mu (B^c) = 0$

Por tanto $\mu (D^c)=0$ y $f=g$ c.p.d. 

{\ } 
 
\textit{b)}
	Si $\{f_n\}$ converge c.p.d. a $f$ y $f=g$ c.p.d. entonces $\{f_n\}$ converge c.p.d. a $g$.
	
Con los conjuntos anteriores sabemos que $\mu (A^c) = 0$, $\mu (D^c) = 0$ y en este caso
\[Si \ x\in D, entonces \ x\in B \implies \mu (B^c) \leq \mu (D^c) = 0\]
Por tanto $\{f_n\}$ converge c.p.d. a $g$.

	%\enunciado{Considerar las siguientes sucesiones $\{f_n\}$ de funciones reales de variable real:}
	\[f_n = \frac{1}{n}\mathcal{X}_{[-n, n]} \hspace{0.5cm} g_n = n^2\mathcal{X}_{[1/(n+1), 1/n]}\]


Estudiar en cada caso la convergencia puntual y comparar $\int lim_{n \rightarrow \infty} f_n$
y $lim_{n \rightarrow \infty} \int f_n$ 

\textit{a)}  $f_n = \frac{1}{n}\mathcal{X}_{[-n, n]}$


$|f_n(x)| \leq \frac{1}{n} \ \forall n\in\mathbb{N}$ por tanto $\{f_n\}$ converge uniformemente a $0$ en $\mathbb{R}$.

$f_n$ es continua c.p.d. y acotada.
\[ \forall n\in \mathbb{N} \hspace{0.5cm} \int_{\R} f_n = 
	\int_{-\infty}^{-n} f_n + \int_{-n}^n f_n + \int_n^{+\infty} f_n = 0 + \int_{-n}^n f_n + 0
\] 
\[ \int_{\R} |f_n| \leq \int_{\R} \frac{1}{n} \mathcal{X}_{[-n, n]} = 
	\frac{1}{n}\int_{\R} \mathcal{X}_{[-n, n]} = \frac{2n}{n} = 2
\]
\[lim_{n\rightarrow \infty} \int f_n = 2 \not = \int lim_{n\rightarrow \infty} f_n = \int 0 = 0\]

{\ }

\textit{b)} $g_n = n^2\mathcal{X}_{[1/(n+1), 1/n]}$

\[\R = ]-\infty, 0[ \cup [0, 5[ \cup [5, +\infty[ \hspace{1cm}
 \forall x\in \left[ \frac{1}{n+1}, \frac{1}{n} \right] \subseteq [0,5[ \hspace{0.5cm} 		
 \{g_n(x)\}\rightarrow 0
\]
\[ x\in[0, 5[ \implies \exists n_o :  x\not\in [0, 5[ \hspace{0.5cm} \forall n\geq m
\]
\[ lim_{n\rightarrow\infty} g_n = 0 \implies \int_{\R} lim_{n\rightarrow\infty} g_n = 0
\]
Comprobamos ahora el límite de la integral
\[ \forall n\in\N \int_{\R} n^2\mathcal{X}_{[\frac{1}{n+1}, \frac{1}{n}]} =
	n^2 \int_{\R}\mathcal{X}_{[\frac{1}{n+1}, \frac{1}{n}]} = \frac{n}{n+1}
\]
Como $lim_{n\rightarrow\infty} \frac{n}{n+1} = 1$, entonces $lim_{n\rightarrow\infty}\int g_n = 1$
	%\enunciado{Sea $E\subseteq \R ^n$ un conjunto medible de medida finita, y sea $\{f_n\}$ una sucesión de funciones medibles en $E$ que converge puntualmente a una función $f$. Supongmos que existe una constante $M\geq 0$ tal que $|\{ f_n \}| \leq M \ \forall n\in\N$. Probar que f es integrable y que
\[ lim\int_E |f-f_n| = 0 \]
Dar un ejemplo mostrando que la hipótesis de medida finita no puede ser suprimida. 
} 

Si $\lambda (E) < +\infty$
\[ \int_E |f| \leq \int_E M = M\lambda (E) < +\infty\]
y por definición $f$ es integrable.

$\{f_n\}$ es una sucesión de funciones medibles que converge puntualmente a $f$ en $E$ y existe una función $g$ integrable en $E$ tal que $\{ |f_n| \}\leq g \ \forall x\in E$. 
Podemos tomar como función $g$ la función constantemente igual a $M$, $g$ es integrable en $E$ ya que $\lambda (E)<+\infty$ (si quitamos la condición de medida finita $g$ no sería integrable).

Por el teorema de la convergencia dominada
\[ lim \int_E |f-f_n| = 0
\]

Si quitamos la condición $\lambda (E)<+\infty$ y cogemos $\{f_n\}$ sucesión de funciones tal que $f_n(x) = 1 \ \forall x \in E$, vemos que $f=lim_{n\rightarrow \infty} f_n$ es una función no integrable, ya que
\[\int_Ef = \int_E1 = \lambda (E) = +\infty \not < +\infty \]
	% 

	%\enunciado{ Sea $E\subseteq \R^n$ un conjunto medible de medida finita y sea $\{ f_n \}$ una sucesión de funciones integrables en $E$ que converge uniformemente a una función $f$. Probar que $f$ es integrable y que
\[ lim \int_E|f-f_n| = 0
\]
Dar un ejemplo mostrando que la hipótesis de medida finita no puede ser suprimida.
} 

\[ f_n \mbox{ converge uniformemente a } f \mbox{ en E y } \lambda (E) < +\infty
\]
\[ \forall \epsilon > 0 \ \exists n_O > 0: n\geq n_0 \ |f(x)-f_n(x)| < \epsilon
\]
\[ f_n(x) - \epsilon \geq f(x) \geq f_n(x) + \epsilon \ \implies \ 
  | f_n(x | \leq | f_n(x) + \epsilon |
\]
Fijando $\epsilon = 1$
\[ \int_E |f(x)dx| \leq \int_E |f_n(x)+1|dx \leq 
   \int_E |f_n(x)|dx + \int_E1dx < +\infty
\]
vemos entonces que $f(x) \ \forall x\in E$ es integrable, y al haber fijado 
$\epsilon , |f(x)-f_n(x)| < 1$ y por el teorema de la convergencia dominada
\[ lim_{n \rightarrow \infty} \int_E |f-f_n| = 0
\]  

{\ }

La hipótesis de medida finita es necesaria:

Definamos en su caso el conjunto $E=\R$, con $f_n = \mathcal{X}_{]0,n]}$ 
% Tenía que era cerrado por la izquierda... (?)
$\{f_n\}$ es una sucesión de funciones integrables, ya que
\[ \int_{\R^+} \frac{1}{n} \mathcal{X}_{]0, n]} = 
   \frac{1}{n} \int_{\R^+} \mathcal{X}_{]0, n]} = \frac{n}{n} = 1 < +\infty
\] 
Sin embargo la función límite no es integrable
\[ f(x) = \left\{
		  \begin{matrix} 
    		      \frac{1}{x} & \mbox{ si }x\in ]0,n[
	          \\ 0        & \mbox{ si } x\not\in ]0, n[      	  
		  \end{matrix} 
		  \right. 
\]


	% 

	% 

	% 

	%\enunciado{Pruébese que la función $e^{x^2}$
 es integrable en el intervalo $[0,1]$ y que
\[ \int_0^1e^{x^2}dx = \sum_{n=0}^{\infty} \frac{1}{(2n+1)n!}
\]} 

\[ \int_0^1|e^{x^2}| \leq \int_0^1e = e < +\infty
\]
Por tanto $e^{x^2}$ es integrable en $[0, 1]$. Definimos $f(y)=e^y:\mathbb{R}\rightarrow\mathbb{R} \hspace{1cm} f\in C^{\infty}(\mathbb{R})$

Desarrollo en serie de Taylor con $a=0$
\[ e^y = \sum_{n=0}^{\infty} \frac{f^{n)}(a)}{n!}(y-a)^n=
\sum_{n=0}^{\infty} \frac{y^n}{n!}
\] y por tanto $e^{x^2}$ se puede expresar como $\sum_{n=0}^{\infty} \frac{x^{2n}}{n!}$

Comprobamos que podemos utilizar el teorema de la convergencia absoluta (*).
\[   \sum_{n=0}^{\infty} \int_0^1 \left| \frac{x^{2n}}{n!} \right| dx
 =   \sum_{n=0}^{\infty} \frac{1}{n!} \left[ \frac{x^{2n+1}}{2n+1} \right]_0^1
 =   \sum_{n=0}^{\infty} \frac{1}{(2n+1) n!} 
\leq \sum_{n=0}^{\infty} \frac{1}{2n^2} < +\infty
\]
Por tanto
\[ \int_0^1 e^{x^2} dx
 = \int_0^1 \sum_{n=0}^{\infty} \left| \frac{x^{2n}}{n!} \right| dx
 = \int_0^1 \sum_{n=0}^{\infty} \frac{x^{2n}}{n!} dx
 =^{(*)} \sum_{n=0}^{\infty} \int_0^1 \frac{x^{2n}}{n!} dx
 = \sum_{n=0}^{\infty} \frac{1}{(2n+1) n!}
\]
	% 


\newpage
\section{T\'ecnicas de integraci\'on}
\subsection{T\'ecnicas de integraci\'on en una variable}
	% 

	% 

	% 

	% 

	% 

	% 

	% 

	
\subsection{T\'ecnicas de integraci\'on en varias variables}
	% 

	% 

	% 

	% 

	% 

	% 

	% 

	% 

	% 

	% 

	% 
		
	% 

	% 


%%%%%%%%%%%%%%%%%%%%%%%%%%%%%%%%%%%%%%%%%%%%%%%%%%%%%%%%%%%%%%%%%%%%%%%%%%%%%%%%%%

% % % % % % % % % % % % % % % % % % % % % % % % % % % % % % % % %
%					 Bibliografía
% % % % % % % % % % % % % % % % % % % % % % % % % % % % % % % % %

\end{document}
