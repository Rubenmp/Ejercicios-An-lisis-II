\begin{ejercicio}{%
Para cada $n \in \N$, sea $f_n:]0,\pi[ \to \R$ la función dada por:
\[f_n(x) = \frac{\sin^2(nx)}{n\sin(x)} \quad (0 < x < \pi)\]
Estudia la convergencia puntual de la sucesión puntual $\{f_n\}$ así como la
convergencia uniforme en intervalos del tipo $]0,a], [a,\pi[$ y $[a,b]$ donde $0 < a< b < \pi$.
}{Pablo Baeyens}

\espacio

\noindent\textbf{$\{f_n\}$  converge puntualmente a 0}

Sea $x \in (0,\pi)$. Para todo $n \in \N$ tenemos:
\[ 0 \leq \frac{\sin^2(nx)}{n\sin(x)} \leq \frac{1}{n\sin(x)} \]
Por tanto, como $\{\frac{1}{n\sin(x)}\} \to 0$, por el teorema del sándwich, la
sucesión converge a 0.

\espacio

\noindent\textbf{La sucesión no converge uniformemente en intervalos de la forma $(0,a]$}

La convergencia uniforme es equivalente a que para cualquier sucesión $\{a_n\}$:

\[\{f_n(a_n) -f(a_n)\} = \left\{\frac{\sin^2(na_n)}{n\sin(a_n)}\right\} \to 0\]

Sea $a_n = \frac{1}{n}$:

\[ \left\{\frac{\sin^2(na_n)}{n\sin(a_n)}\right\}  =
\left\{\frac{\sin^2(1)}{\frac{\sin(\frac{1}{n})}{\frac{1}{n}}}\right\} \to \sin^2(1) \neq 0 \]

\espacio

\noindent\textbf{La sucesión no converge uniformemente en intervalos de la forma $[a,\pi)$}

Sea $a_n = \pi - \frac{1}{n}$:

\[ \left\{\frac{\sin^2(na_n)}{n\sin(a_n)}\right\}  =
\left\{\frac{\sin^2(n\pi -1)}{\frac{\sin(\pi -\frac{1}{n})}{\frac{1}{n}}}\right\} =
\left\{\frac{\sin^2(1)}{\frac{\sin(\frac{1}{n})}{\frac{1}{n}}}\right\} \to \sin^2(1) \neq 0 \]

Ya que $\sin(x) = \sin(\pi-x)$.

\espacio

\noindent\textbf{La sucesión converge uniformemente en intervalos de la forma $[a,b]$}

Por el teorema de Weierstrass
$\exists M >0: \forall x \in [a,b]: \left|\frac{1}{\sin(x)}\right| \leq M$. Por tanto:

\[ \left|\frac{\sin^2(nx)}{n\sin(x)}\right| \leq \frac{M}{n} \]

Como $\{\frac{M}{n}\} \to 0$, la sucesión converge uniformemente en intervalos
de la forma $[a,b]$.

\end{ejercicio}
 
