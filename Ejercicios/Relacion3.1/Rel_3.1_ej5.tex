\enunciado{ Justificar, haciendo uso en cada caso de un conveniente teorema de convergencia, las siguientes igualdades:}

%%%%%%%%%%%%%%%%%%%%%%%%
\textit{d)} $\int_0^{+\infty} \frac{x}{1+e^x}dx = \sum_{n=1}^{\infty} \frac{(-1)^{n+1}}{n^2}$

Expresamos la función inicial de la siguiente forma
\[ \frac{x}{e^x} \left( \frac{1}{1+e^{-x}} \right)\]
Lo expresamos así para poner la segunda parte como una suma infinita de los términos de una progresión geométrica. Esta expresión es válida cuando 
$|-e^{-x}| < 1 \Leftrightarrow x \in ]0, +\infty$, como los puntos de $[0, +\infty]$ donde no se puede expresar la función de esa forma son numerables (solamente falla en el $0$) la integral será la misma.

\[ \frac{x}{e^x} \sum_{n=0}^{+\infty} (-1)^{n+1}e^{-nx} = x \sum_{n=1}^{+\infty} (-1)^{n+1}e^{-nx}
\]
Comprobamos que podemos usar el teorema de la convergencia absoluta (*).   
\[ \sum_{n=1}^{\infty} \int_0^{+\infty} \left| (-1)^{n+1}xe^{-nx} \right| dx
= \sum_{n=1}^{\infty} \int_0^{+\infty} xe^{-nx} dx
\]

Integramos por partes 
\[ \int_0^{+\infty} xe^{-nx}dx \hspace{1cm}
	\begin{bmatrix}
	f  = x    &  dg = e^{-nx}dx \\
	df = dx   &   g = \frac{e^{-nx}}{-n}
	\end{bmatrix}
\]
Cambio de variable
\[ -\frac{xe^{-nx}}{n} + \frac{1}{n} \int_0^{+\infty}e^{-nx}dx \hspace{1cm} 
	\begin{bmatrix}
	u = -nx   \\
	du = -ndx 
	\end{bmatrix}
\]
\[ -\frac{xe^{-nx}}{n} + \frac{1}{n^2} \int_0^{+\infty}e^udx 
=  \left[ -\frac{e^{-nx}(nx+1)}{n^2} \right]_0^{+\infty} 
= \frac{1}{n^2}
\]
Como $\sum_{n=0}^{+\infty} \frac{1}{n^2}$ converge podemos intercambiar el sumatorio con la integral.

\[ \int_0^{+\infty} \frac{x}{1+e^x}dx 
 = \int_0^{+\infty} \sum_{n=1}^{+\infty} (-1)^{n+1}xe^{-nx}   dx
 =^{(*)} \sum_{n=1}^{+\infty} (-1)^{n+1}  \int_0^{+\infty} xe^{-nx} dx
 = \sum_{n=1}^{+\infty} \frac{(-1)^{n+1}}{n^2}
\]




%%%%%%%%%%%%%%%%%%%%

\textit{e)} $\int_0^{+\infty} \frac{e^{-ax}}{1+e^{-bx}}dx = \sum_{n=0}^{+\infty} \frac{(-1)^n}{a+nb}$
$(a,b>0)$ y deducir que 
\[ \sum_{n=0}^{+\infty} \frac{(-1)^n}{2n+1} = \frac{\pi}{4} \hspace{0.5cm}
   \sum_{n=0}^{+\infty} \frac{(-1)^n}{n+1}  = ln(2) 
\]

Empecemos expresando el integrando como una constante y la suma infinita de una progresión geométrica.
Como el en caso anterior esta expresión es válida $\forall x\in ]0, +\infty[$, al diferenciarse con
la original en un número finito de puntos las integrales son iguales.
\[ \frac{e^{-ax}}{1+e^{-bx}}
 = e^{-ax}\frac{1}{1+e^{-bx}}
 = e^{-ax} -\sum_{n=0}^{+\infty} (-e^{-bx})^n
 = -\sum_{n=0}^{+\infty} (-1)^n e^{-ax-bnx}
\]

¿Podemos usar el teorema de la convergencia absoluta?
\[ \sum_{n=0}^{+\infty} \int_0^{+\infty} | (-1)^ne^{-ax-bnx} |
 = \sum_{n=0}^{+\infty} \int_0^{+\infty} e^{-ax-bnx}
 = -\sum_{n=0}^{+\infty} \left[ \frac{e^{-ax-bnx}}{a+bn} \right]_0^{+\infty} 
 = -\sum_{n=0}^{+\infty} \frac{1}{a+bn}
\]

Como el sumatorio anterior diverge no podemos usar el teorema de la convergencia absoluta, probemos
con el teorema de la convergencia dominada.

Definimos nuestra sucesión de funciones integrables $f_n(x) = \sum_{i=0}^{n} \frac{(-1)^i}{e^{ax+bix}}$.
Es claro que
\[ \{f_n(x)\} \to \sum_{i=0}^{+\infty} \frac{(-1)^i}{e^{ax+bix}}
\]



Vamos a intentar acotar los elementos de la sucesión en valor absoluto
\[ |f_n(s)| \leq \sum_{i=0}^{E(n)} \frac{(-1)^i}{e^{ax+bix}} \hspace{1cm} \forall n,x\in\R
\]







\textbf{Deducciones}

- Caso $a=1m¡,b=2$
\[ \sum_{n=0}^{+\infty} \frac{(-1)^n}{1+2n}
 = \int_0^{+\infty} \frac{e^{-x}}{1+e^{-2x}}dx \hspace{1cm}
	\begin{bmatrix}
	u  = e^x  \\
	du = e^x dx
	\end{bmatrix}
\]
\[ = \int_0^{+\infty} \frac{1}{\left( \frac{1}{u^2} +1 \right) u^2}du 
   = \int_0^{+\infty} \frac{1}{1+s^2} ds
    \hspace{1cm}
	\begin{bmatrix}
	s  = \frac{1}{u} \\
	ds = -\frac{1}{u^2}du	
	\end{bmatrix}
\]
Integramos y deshacemos los cambios de variable
\[ -tan^{-1}(s) \hspace{0.5cm} -tan^{-1}\left( \frac{1}{u}\right) \hspace{0.5cm}
   -tan^{-1} (e^{-x})
\]
Por tanto
\[ \int_0^{+\infty} \frac{e^{-x}}{1+e^{-2x}}dx
 = \left[ -tan^{-1} (e^{-x}) \right]
 = -(0-\frac{\pi}{4}) = 	\frac{\pi}{4}
\]


- Caso $a=b=1$
\[ \sum_{n=0}^{+\infty} \frac{(-1)^n}{n+1} 
 = \int_0^{+\infty} \frac{e^{-x}}{1+e^{-x}} dx \hspace{1cm}
	\begin{bmatrix}
	u  = -x \\
	du = -dx
	\end{bmatrix}
\]
\[ = - \int_0^{+\infty} \frac{e^u}{1+e^u} du 
   = - \int_0^{+\infty} \frac{ds}{s} \hspace{1cm}
	\begin{bmatrix}
	s = e^u + 1 \\
	ds = e^u du
	\end{bmatrix}
\]
Hacemos los correspondientes cambios de variable
\[ -ln(s) \hspace{0.5cm} -ln(e^u+1) \hspace{0.5cm} -ln(e^{-x}+1)
\]
\[ \int_0^{+\infty} \frac{e^{-x}}{1+e^{-x}} dx 
 = \left[ -ln(e^{-x}+1) \right]_0^{+\infty} = -ln(2)
\]
