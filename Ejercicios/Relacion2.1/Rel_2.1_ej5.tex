\enunciado{5. Probar que M es la mayor $\sigma$ -\'algebra que contiene los intervalos acotados y sobre la que $\lambda ^*$ es aditiva.}



Supongamos que existe otra $\sigma$ -\'algebra $N$ que contiene los intervalos acotados y sobre la que $\lambda ^*$ es aditiva. Terminaremos demostrando que en ese caso $N \subseteq M$.

Recordemos que $M=\{B \cup Z : B\in \mathfrak{B}, \lambda ^*(Z)=0\} \subseteq C_{\mathbb{R},\lambda }$

La $\sigma$-subaditividad nos decía:
\[\lambda ^*(A\cup B) \leq \lambda ^* (A)+ \lambda ^*(B)\]

Llamaremos $\lambda '=\lambda	^* / N$

Por ser una $\sigma$-\'algebra $\Omega \in N$, al estar hablando de intervalos $\Omega = \mathbb{R}$

Sea $E\subseteq \mathbb{R}$, cogemos un conjunto arbitrario $A\subseteq \mathbb{R}$.
Sabemos, en virtud de la propiedad de regularidad de la medida exterior (Prop. 2.1.10), que existe un boreliano $B$

\[ B:A\subseteq B, \lambda '(A)=\lambda '(B) \]

$\lambda '(A)=\lambda '(B)$,  usando la propiedad de que $N$ contiene los intervalos acotados podemos usar la $\sigma-$aditividad. 
$B\cap E, B\cap E^c \in N$ 

$\lambda '(B) = \lambda'\left( (B\cap E) \cup (B\cap E^c) \right) $
$\geqslant\lambda '(A\cap E) + \lambda'(A\cap E^c) \geqslant \lambda '(A)$

Por tanto $E \in C_{\mathbb{R},\lambda '}$ lo que es equivalente a $E \in M$
$\implies N \subseteq M$
 
