\enunciado{
Pruébese que la función $\frac{sen(x)}{x}$ es integrable en el intervalo $[0,1]$ y que
$$\int_{0}^{1}\frac{sen(x)}{x}dx = \sum_{n=0}^{\infty}\frac{(-1)^n}{(2n+1)(2n+1)!}$$}


Sabemos que el límite de la función en la abscisa $x=0$ es 1, por L'Hôpital. También podemos sacar que $\frac{sen(x)}{x}$ es una función decreciente en ese intervalo si estudiamos las derivadas, y además positiva, porque los dos términos lo son. Por lo tanto, sabemos que esta función esta acotada por 1 en el intervalo $[0,1]$. Siendo una función continua, concluimos que es integrable.

La idea de la igualdad va a ser aplicar el Teorema de la convergencia absoluta. Definiremos la sucesión de funciones $f_n$ como la expresión por el desarrollo de Taylor del $sen(x)$, transformándolo para obtener una expresión de $\frac{sen(x)}{x}$:
$$sen(x) =\sum_{n=0}^{\infty}\frac{(-1)^n}{(2n+1)!}x^{2n+1} \hspace{0.5cm} \forall x \in (0,1) \Rightarrow \frac{sen(x)}{x} =\sum_{n=0}^{\infty}\frac{(-1)^n}{(2n+1)!}x^{2n}\hspace{.5cm} \forall x \in (0,1) $$

La igualdad se da en el intervalo abierto, pero como se distingue del cerrado en un conjunto numerable de puntos, podemos estudiar la integral en este intervalo, que será la misma.
Definimos entonces $f_n : (0,1) \rightarrow \mathds{R}$:

$$f_n(x) = \frac{(-1)^n}{(2n+1)!}x^{2n}$$

Estas funciones son continuas, crecientes, conforman una serie de potencias y nos permiten bastante libertad para operar con ellas. Para usar el Teorema de la convergencia absoluta, vamos a probar que la suma del valor absoluto de las integrales de las $f_n$ en $(0,1)$ está acotada:

$$\sum_{n=0}^{\infty}\int_{0}^{1}|f_n(x)|dx = \sum_{n=0}^{\infty}\int_{0}^{1}\frac{1}{(2n+1)!}x^{2n}dx \leq \sum_{n=0}^{\infty}\int_{0}^{1}\frac{1}{(2n+1)!}dx = \sum_{n=0}^{\infty}\frac{1}{(2n+1)!} \leq \sum_{n=0}^{\infty}\frac{1}{(n+1)^2}$$

Y como esta última serie converge, sabemos que la primera también lo hace. Aplicando entonces el teorema, tenemos que:

$$\int_{0}^{1}\left(\sum_{n=0}^{\infty}f_n(x)\right)dx = \sum_{n=0}^{\infty}\left(\int_{0}^{1}f_n(x) dx\right)$$

Sustituimos $\frac{sen(x)}{x}$ con la sumatoria, usando la igualdad antes mencionada. Además, la integral de la serie de potencias de la derecha podemos hacerla por Riemann, o usando la regla de Barrow (que aparece en el siguiente tema), hallamos fácilmente la integral de cada $f_n$ en $(0,1)$, sabiendo que la función cuya derivada es $f_n$ es $\frac{(-1)^n}{(2n+1)(2n+1)!}x^{2n+1}$:
$$\int_{0}^{1}\frac{sen(x)}{x}dx = \sum_{n=0}^{\infty}\left(\int_{0}^{1}f_n(x) dx\right) =\sum_{n=0}^{\infty}\frac{(-1)^n}{(2n+1)(2n+1)!} $$
